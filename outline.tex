\PassOptionsToPackage{unicode=true}{hyperref} % options for packages loaded elsewhere
\PassOptionsToPackage{hyphens}{url}
%
\documentclass[]{article}
\usepackage{lmodern}
\usepackage{amssymb,amsmath}
\usepackage{ifxetex,ifluatex}
\usepackage{fixltx2e} % provides \textsubscript
\ifnum 0\ifxetex 1\fi\ifluatex 1\fi=0 % if pdftex
  \usepackage[T1]{fontenc}
  \usepackage[utf8]{inputenc}
  \usepackage{textcomp} % provides euro and other symbols
\else % if luatex or xelatex
  \usepackage{unicode-math}
  \defaultfontfeatures{Ligatures=TeX,Scale=MatchLowercase}
\fi
% use upquote if available, for straight quotes in verbatim environments
\IfFileExists{upquote.sty}{\usepackage{upquote}}{}
% use microtype if available
\IfFileExists{microtype.sty}{%
\usepackage[]{microtype}
\UseMicrotypeSet[protrusion]{basicmath} % disable protrusion for tt fonts
}{}
\IfFileExists{parskip.sty}{%
\usepackage{parskip}
}{% else
\setlength{\parindent}{0pt}
\setlength{\parskip}{6pt plus 2pt minus 1pt}
}
\usepackage{hyperref}
\hypersetup{
            pdfborder={0 0 0},
            breaklinks=true}
\urlstyle{same}  % don't use monospace font for urls
\setlength{\emergencystretch}{3em}  % prevent overfull lines
\providecommand{\tightlist}{%
  \setlength{\itemsep}{0pt}\setlength{\parskip}{0pt}}
\setcounter{secnumdepth}{0}
% Redefines (sub)paragraphs to behave more like sections
\ifx\paragraph\undefined\else
\let\oldparagraph\paragraph
\renewcommand{\paragraph}[1]{\oldparagraph{#1}\mbox{}}
\fi
\ifx\subparagraph\undefined\else
\let\oldsubparagraph\subparagraph
\renewcommand{\subparagraph}[1]{\oldsubparagraph{#1}\mbox{}}
\fi

% set default figure placement to htbp
\makeatletter
\def\fps@figure{htbp}
\makeatother


\date{}

\begin{document}

\hypertarget{header-n0}{%
\section{Tic-Tac-Toe like a Pro}\label{header-n0}}

\textbf{Vortrag zum Thema Spielbäume}

\hypertarget{header-n20}{%
\subsection{Agenda}\label{header-n20}}

\begin{itemize}
\item
  Motivation
\item
  Terminologie
\item
  MinMax
\item
  AlphaBeta
\end{itemize}

\hypertarget{header-n27}{%
\subsection{Motivation}\label{header-n27}}

\hypertarget{header-n34}{%
\subsubsection{Donald Michies "Menace"}\label{header-n34}}

\begin{itemize}
\item
  1963
\item
  \ldots{}
\end{itemize}

\hypertarget{header-n1199}{%
\subsubsection{Ziel dieses Vortrages}\label{header-n1199}}

\begin{itemize}
\item
  Verständnis für formale Betrachtung von Spielen
\item
  Schritte des MinMax-Algorithmus nachvollziehen können (in der Lage
  sein es zu implementieren)
\item
  Notwendigkeit der Alpha-Beta-Kürzung verstehen und anpassung des
  Algorithmusses verstehen
\item
  weitere?
\end{itemize}

\hypertarget{header-n64}{%
\subsection{Terminologie}\label{header-n64}}

\hypertarget{header-n73}{%
\subsubsection{Einordnung "Spiel"}\label{header-n73}}

\begin{itemize}
\item
  {[}Grafik Spiele{]}
\item
  Definition: Spiel S ≔ (R, a, F )

  \begin{itemize}
  \item
    R Menge legaler Spielzüge
  \item
    k0 ∈ K Anfangskonfiguration
  \item
    F ⊂ K Menge der Endkonfigurationen
  \item
    zwei Spieler: der Max und der Min
  \end{itemize}
\item
  K Menge aller zulässigen Konfigurationen
\end{itemize}

\hypertarget{header-n203}{%
\subsubsection{Spielzüge, Folgekonfiguration}\label{header-n203}}

\begin{itemize}
\item
  R ≔ \{r ∶ u → v ∣ u, v ∈ K, r legal \}

  \begin{itemize}
  \item
    wenn v = r(u) , dann heißt v Folgekonfiguration von u (bezüglich r )
  \item
    analog: u Elternkonfiguration von v (bezüglich r )
  \end{itemize}
\item
  R k ≔ \{r ∈ R ∣ r anwendbar auf k\}
\item
  Menge der Folgekonfigurationen N (k) ≔ \{r(k) ∣ r ∈ R k \} ⊆ K

  \begin{itemize}
  \item
    N(k f ) = ∅, k f ∈ F
  \end{itemize}
\item
  K induktiv über Folgekonfigurationen definierbar: (1) a ∈ K (2) ∀k ∈ K
  ∶ N(k) ⊂ K
\item
  {[}Grafik Spielzüge{]}
\end{itemize}

\hypertarget{header-n305}{%
\subsubsection{Spielbaum (ehemals Spielgraph)}\label{header-n305}}

\begin{itemize}
\item
  Spielgraph ist gerichteter Graph G S ≔ (V , E)

  \begin{itemize}
  \item
    Knoten V = K
  \item
    Kanten E = ⋃ \{(u, v) ∣ v ∈ N(u)\}
  \end{itemize}
\item
  G ist Baum mit Wurzel a , Blättern F
\item
  Spielverlauf mit n Zügen ist Pfad p subsequenter Folgekonfigurationen

  \begin{itemize}
  \item
    p = (k 0 = a, k 1 , ... , k n ∈ F )
  \end{itemize}
\item
  theoretisches Hilfskonstrukt
\item
  {[}Grafik: TicTacToe Spielgraph{]}
\end{itemize}

\hypertarget{header-n382}{%
\subsubsection{Suchbaum (ehemals Spielbaum)}\label{header-n382}}

\begin{itemize}
\item
  Spielbaum ist gerichteter Graph B S ≔ (V , E, l, k 0 )

  \begin{itemize}
  \item
    Tiefe l ∈ N, l ≥ 1
  \item
    Wurzelknoten k 0 ∈ K
  \item
    Knoten V = \{v = (k 0 , ... , k n ) ∣ ∀k i ∈ v ∶ ∃r ∈ R k ∶ k i =
    r(k i−1), k i ∈ K, n ≤ l\}

    \begin{itemize}
    \item
      entspricht (Teil-)Pfaden der maximalen Länge l , ausgehend von k 0
      im Spielgraph
    \end{itemize}
  \item
    Kanten E = \{((k 0 , ... , k i ), (k 0 , ... , k i , k i+1 )) ∣ ∃r ∈
    R k ∶ r(k i) = k i+1 \}
  \end{itemize}
\item
  praktisch in Implementierung genutzt
\end{itemize}

\hypertarget{header-n461}{%
\subsection{MinMax}\label{header-n461}}

\hypertarget{header-n464}{%
\subsubsection{Geschichte}\label{header-n464}}

\begin{itemize}
\item
  populär als Suchalgorithmus für Spielstrategie in Schachprogrammen
\item
  1912 erstmals von Ernst Zermelo erwähnt
\item
  1940er Bedeutung betont (Shannon, Turing)
\end{itemize}

\hypertarget{header-n490}{%
\subsubsection{wir brauchen\ldots{}}\label{header-n490}}

\begin{itemize}
\item
  Nutzenfunktion f: F → N, die Endkonfigurationen mit einer Zahl
  bewertet (der Einfachheut natürliche Zahlen)
\item
  zwei Typen von Knoten: Min- und Max- je nach dem wer am Zug ist, je
  nach dem ob Minimiert oder Maximiert werden soll

  \begin{itemize}
  \item
    {[}Grafik: Bildsprache Min-/ Max-Knoten{]}
  \item
    bei uns: alterniert mit jedem Halbzug
  \end{itemize}
\end{itemize}

\hypertarget{header-n660}{%
\subsubsection{Algo}\label{header-n660}}

\begin{itemize}
\item
  jedem Knoten wird ein MinMax-Wert zugeordnet

  \begin{itemize}
  \item
    entspricht Nutzen dieser Konfiguration für Max
  \item
    der MinMax-Wert eines Knoten ergibt sich immer aus seinen Kindern je
    nach dem, ob es ein Min- oder Max-Knoten gibt, bzw. entspricht f(k),
    wenn k Endkonf.

    \begin{itemize}
    \item
      Min-Knoten: Annahme: Min ist an der Reihe und spielt optimal, d.h.
      Annahme 2: bester Zug für Min ist schlechtester für Max →
      kleinster Wert für Max anzunehmen (so zu sagen worst case)
    \item
      Max-Knoten: Max ist dran und nimmt besten Zug → höchster Wert
    \end{itemize}
  \item
    schließlich: es wird Zug gewählt, der Max' Nutzen maximiert, bzw,
    der Zug zum Kindknoten mit dem selben Wert, wie dem aktuellen
    Wurzelknoten
  \end{itemize}
\item
  {[}Formel: MinMax, S. 209{]}
\item
  {[}Grafik: MinMax TicTacToe{]}
\item
  {[}Grafik: abstrakter: dreieckige Min-/Max-Knoten{]}
\item
  Rekursion

  \begin{itemize}
  \item
    steigt Spielpfade entlang des Baumes bis zu Blättern ab
  \item
    Knoten werden quasi inialisiert: -∞ für Max- und ∞ für Min-Knoten
  \item
    entspricht vollständiger Tiefensuche
  \item
    Aufstieg: tatsächliche Werte werden, basierend auf Kindern
    gespeichert
  \end{itemize}
\end{itemize}

\hypertarget{header-n1111}{%
\subsubsection{Komplexität}\label{header-n1111}}

\begin{itemize}
\item
  m maximale Tiefe des Baumes
\item
  b Verzweigungsgrad an jedem Knoten (mögliche Züge)
\item
  Zeit: O(b\^{}m)
\item
  Speicher: O(bm) wenn alles gleiczeitig gehalten wird, O(m), wenn
  nacheinander, d.h. für einen Pfad
\item
  Problem: Baumsuche: Anzahl möglicher Spielstände wächst exponentiell
  zur Anzahl Züge, real nicht brauchbar
\end{itemize}

\hypertarget{header-n526}{%
\subsection{Alpha-Beta}\label{header-n526}}

\begin{itemize}
\item
  Pruning = Kürzung, um weniger Konfigurationen zu durchsuchen

  \begin{itemize}
  \item
    gibt das selbe Ergebnis wie std-Min-Max zurück
  \item
    Konfigurationen, die Entscheidung nicht beeinflussen, werden nicht
    betrachtet
  \end{itemize}
\end{itemize}

\hypertarget{header-n780}{%
\subsubsection{Geschichte}\label{header-n780}}

\begin{itemize}
\item
  Optimierung von Minmax
\item
  1956 in Dartmouth vorgestellt (McCarthy)
\item
  1961 erstmals in Schachprogramm eingesetzt
\item
  1975 Korrektheit bewiesen (Knuth, Moore)
\item
  weitere Verbesserungen u. Varianten folgten
\end{itemize}

\hypertarget{header-n943}{%
\subsubsection{Wie brauchen\ldots{}}\label{header-n943}}

\begin{itemize}
\item
  da tiefensuche, stats pfadweise Betrachtung, pro Pfad muss
  alpha/beta-Par gehalten werden

  \begin{itemize}
  \item
    praktisch bekommet jeder Knoten einen alpha/ beta Wert
  \item
    alpha: bisher bester wert im pfad
  \item
    beta: bisher schlechtester wert im pfad
  \end{itemize}
\end{itemize}

\hypertarget{header-n1257}{%
\subsubsection{Algo}\label{header-n1257}}

\begin{itemize}
\item
  alpha und beta werden nach jeder untersuchten konfiguration
  aktualisert

  \begin{itemize}
  \item
    initial wird alpha = -ue und beta = +ue gesetzt iwe minmax
  \item
    alpha wird max-knoten maximiert: max(a-1, v)
  \item
    beta wird an minknoten minimiert: min(b-1,v)
  \end{itemize}
\item
  kinder werden gekürzt = rekursion vor erreichen der endkonf.
  terminiert

  \begin{itemize}
  \item
    an Max-Knoten k: α-Cut: es wird der größte Wert der Kinder gesucht

    \begin{itemize}
    \item
      sobald β eines Kindes kleiner gleich als α des k
    \end{itemize}
  \item
    an Min-Noten k: β-Cut: es wird der kleinste Wert der Kinder gesucht

    \begin{itemize}
    \item
      sobald α eines Kindes größer gleich als β des k
    \end{itemize}
  \end{itemize}
\end{itemize}

\hypertarget{header-n989}{%
\subsubsection{Komplexität}\label{header-n989}}

\begin{itemize}
\item
  Minmax: O(b\^{}m)
\item
  AlphaBetta O(b\^{}(m/2) bei optimaler Sortierung (Heuristik)

  \begin{itemize}
  \item
    d.h. in gleicher zeit doppelt so tiefen baum
  \item
    theoretisches limit, wenn immer sofort gekürzt werden kann
  \end{itemize}
\item
  Zufällige Sortierung der Kinde: etwa O(b\^{}(3/4m))
\end{itemize}

\hypertarget{header-n1433}{%
\subsection{unvollständige Echtzeitentscheidung}\label{header-n1433}}

\hypertarget{header-n1436}{%
\subsubsection{verbesserung}\label{header-n1436}}

\begin{itemize}
\item
  kriteirum, wann nicht mehr weiter expandieren
\item
  zusätzlich zu Nutzenfunktion für Endkonf.: Bewertungsfunktion für
  Konfs, die nicht Endkonf sind (Heuristik)

  \begin{itemize}
  \item
    vergleichbar mit Kostenschätzung bei A* Suche
  \end{itemize}
\item
  mit alphabeta kombinierbar
\item
  stärke des agenten hängt von genauigkeit der Heuristik und bewertung
  des nutztens einer knotenexpansion ab
\end{itemize}

\hypertarget{header-n1470}{%
\subsubsection{Heuristik}\label{header-n1470}}

\begin{itemize}
\item
  schätzt den nutzen für spieler in bestimmter konfiguration ab
\item
  anforderungen

  \begin{itemize}
  \item
    muss mindestens gleiche ordnung, wie nutzenfunktion erzeugen
  \item
    muss schnell sein
  \item
    sollte bei nicht endzuständen nah an tatsächlichen chancen sein
    (chance, auch wenn kein zufallsspiel da information durch
    tiefenbeschränkung unvollständig wird)
  \end{itemize}
\item
  setzt sich aus verschiedenen merkmalen des zustands zusammen, die sich
  numerisch beschreiben lassen
\item
  {[}bsp tictactoe{]}
\end{itemize}

\hypertarget{header-n1544}{%
\subsubsection{wann abbrechen}\label{header-n1544}}

\begin{itemize}
\item
  immer wenn endkonf erreicht ist natürlich
\item
  bspw. feste tiefe → problematisch (bsp. schach)

  \begin{itemize}
  \item
    mindest suchtiefe, dann abbrechen, wenn bewertung ruht
  \end{itemize}
\item
  metaschließen: wann lohnt es sich eine berechnung anzustellen?
\end{itemize}

\hypertarget{header-n792}{%
\subsection{Ausblick}\label{header-n792}}

\begin{itemize}
\item
  Min spielt nicht optimal

  \begin{itemize}
  \item
    MinMax nur beste Strategie, wenn Min wirklich optimal spielt
  \item
    wenn nicht: dann noch besser für Max
  \item
    aber: es gibt dann ggf. noch bessere Strategien
  \end{itemize}
\item
  Nicht-nullsummenspiel

  \begin{itemize}
  \item
    jeder spieler hat eigene nutzenfunktion (die beiden bekannt ist)
  \end{itemize}
\item
  mehr als 2 spieler

  \begin{itemize}
  \item
    nicht Min-/Max-Halbzüge, sondern A,B,C,\ldots{}
  \item
    nicht ein MinMaxVal sondern Vektor mit Max-Werte jedes Spielers
  \item
    warum kein vektor bei 2?

    \begin{itemize}
    \item
      nullsummenspiel gewinn ist verlust des anderen
    \item
      negamax
    \end{itemize}
  \item
    Problem: Spielstrategie mit Allianzen
  \end{itemize}
\item
  zugreihenfolge

  \begin{itemize}
  \item
    vorsortieren und Killerzüge finden ("Killerzugheuristik")
  \item
    dynamisch: zuerst züge probieren, die vorher gut waren

    \begin{itemize}
    \item
      situationen können wieder auftreten → hashtabelle (allerdings für
      alle knoten wiederum zu aufwendig) für interessante konf.s
    \end{itemize}
  \end{itemize}
\item
  verbesserung mit iterativer tiefensuche

  \begin{itemize}
  \item
    hilft bei sortierung der zugreihenfolge
  \item
    begrenzte zeit möglichst gut ausnutzen
  \end{itemize}
\item
  Spiele mit Zufall

  \begin{itemize}
  \item
    dritte art knoten: zufallsknoten
  \item
    erwartungswert statt minmax-wert
  \end{itemize}
\item
  "Maximin" in der Philosophie

  \begin{itemize}
  \item
    US-amerikanischer Philosoph: John Rawles: "A Theory of Justice",
    1971, belebte politische Philosophie wieder
  \item
    Gedankenexperiment zur Schaffung eines neuen Gesellschaftsvertrages

    \begin{itemize}
    \item
      Vertragspartner befinden sich in hypotetischem Urzustand hinter
      Schleier des Nichtwissens und entscheiden pber
      Gerechtigkeitsprinzipien
    \item
      wissen nicht, wo/wer sie sein werden
    \end{itemize}
  \item
    Grundsätze der Gerechtigkeit
  \item
    1) Jeder Mensch hat das gleiche Recht auf das umfangreichste
    Gesamtsystem gleicher Grundfreiheiten, das für alle möglich ist.
  \item
    2) Soziale und wirtschaftliche Ungleichheiten müssen folgendermaßen
    beschaffen sein:

    \begin{itemize}
    \item
      (a) sie müssen unter der Einschränkung des gerechten
      Spargrundsatzes den am wenigsten Begünstigten den größtmöglichen
      Vorteil bieten, und

      (b) sie müssen mit Ämtern und Positionen verbunden sein, die allen
      gemäß fairer Chancengleichheit offen stehen.
    \end{itemize}
  \item
    Vorrang von 1 vor 2 sowie ein Vorrang des Prinzips fairer
    Chancengleichheit (b) vor dem Differenzprinzip (a)
  \item
    MaxiMin bedeutet nun, dass die entscheidende Person sich für die
    Alternative entscheidet, die das minimale denkbare Ergebnis
    maximiert. Rawls wählt damit das konservative Prinzip der
    Risikominimierung. (Risikoaversion)
  \end{itemize}
\end{itemize}

\hypertarget{header-n597}{%
\subsection{Anhang}\label{header-n597}}

\begin{itemize}
\item
  Github-Link
\item
  Quellen
\end{itemize}

\end{document}
