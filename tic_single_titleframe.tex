\documentclass[tikz]{standalone}
\usepackage{forest}
\usepackage{calc}


\forestset{
  make tab/.style args={#1:#2:#3/#4:#5:#6/#7:#8:#9}{%
    content={%
      \tabcolsep=.6\tabcolsep
      \begin{tabular}{p{\widthof{x}}|p{\widthof{x}}|p{\widthof{x}}}
              #1 & #2 & #3\\
        \hline#4 & #5 & #6\\
        \hline#7 & #8 & #9
      \end{tabular}}},
  label position r/.initial=right,
  label position b/.initial=below
}

\definecolor{g}{HTML}{eeeeee}
\definecolor{a}{HTML}{f3d9a7}
\definecolor{b}{HTML}{d9f3a7}
\definecolor{c}{HTML}{a7c7f3}

\begin{document}
\begin{forest}
  TTT/.style args={#1:#2}{
    make tab/.expanded=\forestove{content},
    label={\pgfkeysvalueof{/forest/label position #1}:$#2$}
  },
  TTT*/.style={
    make tab=::/::/::,
    content/.expand once=%
    \expandafter\vphantom\expandafter{\romannumeral-`0\forestov{content}},
    draw=none,
    append after command={(\tikzlastnode.north) edge (\tikzlastnode.south)},
    for descendants={before computing xy={l*=1.2}},
  },
  th/.style=thick,
  for tree={node options=draw, inner sep=+0pt, parent anchor=south, child anchor=north}
%
[::x/x:o:/::o, TTT=r:]
\end{forest}
\end{document}